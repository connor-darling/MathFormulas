\documentclass{report}

\input{preamble}
\input{macros}
\input{letterfonts}

\title{\Huge{Math 3B}\\Take Home Quizzes}
\author{\huge{Connor Darling}}
\date{}

\begin{document}

\maketitle
\newpage% or \cleardoublepage
% \pdfbookmark[<level>]{<title>}{<dest>}
\pdfbookmark[section]{\contentsname}{toc}
\tableofcontents
\pagebreak

\setcounter{chapter}{2}
\chapter{Techniques of Integration}
\setcounter{section}{1}
\section{Trignometric Integrals}
\section{Trignometric Substitution}
\section{Partial Fractions}
\setcounter{section}{5}
\section{Numerical Integration}
\section{Improper Integrals}

\chapter{Introduction to Differential Equations}
\setcounter{section}{2}
\section{Seperable Differential Equations}

\chapter{Sequences and Series}


\section{Quiz 8 - Sequences}
\newpage

\qs{}{Find a formula for $a_n$, the general term of the following sequence, assuming the
initial index is 1 and the pattern suggested by the first few terms continues.

  \begin{equation*}
    \left\{
      \frac{4}{9}, \frac{-6}{27}, \frac{8}{81}, \frac{-10}{243}, \dots
    \right\}
  \end{equation*}
}

\pf{Answer to question 1}{
  \begin{equation*}
    \left\{ a_n \right\} = 
    \left\{
      \frac{4}{9}, \frac{-6}{27}, \frac{8}{81}, \frac{-10}{243}, \dots
    \right\} =
    \left\{
      \frac{2(n+1)(-1)^{(n+1)}}{3^{(n+1)}}
    \right\} 
  \end{equation*}

  The denominator increase by multiple of $\frac{1}{3}$ each time which is the same as 
  an incrementing power $3^{n}$, but since the sequence begins with a denominator 9
  we need to raise 3 to a power which will be the starting index plus one:
  $3^{(n+1)}$ to begin our sequence at 9. \\

   As for the numerator, each succeeding element
  in the list varies from positive, negative, positive, negative, \dots and so on. 
  This tells us 
  that we are going to need to multiply by a negative one each time that is raised to 
  an even integer to get a positive number, and an odd integer to get a negative number
}

\qs{}{Find the limit of the following sequence.
  \begin{equation*}
    \left\{
      \frac{e^{-n}}{\sin \left( \frac{1}{n} \right)}
    \right\}
  \end{equation*}
}

\pf{Answer to question 2}{
  \begin{equation*}
    \left\{
      \frac{e^{-n}}{\sin \left( \frac{1}{n} \right)}
    \right\} \\

  \end{equation*}

  \begin{equation*}
  \lim_{n\to\infty} \left( \frac{e^{-n}}{\sin \left (\frac{1}{n} \right)} \right)
  \end{equation*} \\

  By Squeeze Theorem:

  \begin{equation*}
    \frac{e^{-n}}{-1} \leqq \frac{e^{-n}}{\sin \left(\frac{1}{n} \right)} \leqq \frac{e^{-n}}{-1}
  \end{equation*} \\

  \begin{equation*}
  \lim_{n\to\infty} \abs[\Big]{\frac{e^{-n}}{-1}} = 0 \;\;\;\;and\;\;\;\; \lim_{n\to\infty} \abs[\Big]{\frac{e^{-n}}{1}} = 0
  \end{equation*} \\

  \begin{equation*}
    \therefore \; \lim_{n\to\infty} \left( \frac{e^{-n}}{\sin \left (\frac{1}{n} \right)} \right) = 0
  \end{equation*} \\

}


\newpage
\qs{}{Determine whether the following sequence is (eventually) increasing, decreasing,
  or neither. If the behavior is only true eventually, be sure to state this.
  \begin{equation*}
    \left\{
      \frac{n!}{2^n}
    \right\}
  \end{equation*}
} % END OF SECTION 

\end{document}
