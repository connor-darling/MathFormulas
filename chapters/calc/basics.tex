\chapter{Calculus I}

\noindent This chapter will embody most of calculus I topics, excluding 
basic integration which will be saved for the following chapter calculus II

% !
\section{Definition of the Limit}
\noindent \textbf{Limit}\\
The limit of a function is the value the function approaches at a given value of
$x$, regardless of whether the function actually reaches that value.

\begin{framed}
  \noindent \textbf{Example:} Limit
  \begin{equation*}
    \lim_{x \to 2} \frac{x^2 - 4}{x - 2}
  \end{equation*}
  \noindent This represents the limit of the function $(x^2 - 4)/(x - 2)$ as $x$ 
            approaches 2.
\end{framed}

\noindent \textbf{One-sided limits}\\
The one-sided limits are the left- and right-hand limits. The left-hand limit is 
the limit of the function as we approach from the left side (or negative side), 
whereas the right-hand limit is the limit of the function as we approach from 
the right side (or positive side).

\begin{framed}
  \noindent \textbf{Example:} One Sided Limit
  \begin{equation*}
    \lim_{x \to 0^+} \frac{1}{x}
  \end{equation*}
  \noindent This represents the limit of the function $1/x$ as $x$ approaches 0 
  from the right (or positive direction).\\\\ 

  \noindent \textbf{Another Example:} One Sided Limit
  \begin{equation*}
    \lim_{x \to 0^-} \frac{1}{x}
  \end{equation*}
  \noindent This represents the limit of the function $1/x$ as $x$ approaches 0 
  from the left (or negative direction).
\end{framed}

\noindent \textbf{General Limit}\\
The general limit exists at a point $x=c$ if
\begin{enumerate}
  \item the left-hand limit exists at $x=c$,
  \item the right-hand limit exists at $x=c$, and
  \item those left and right-hand limits are equal to one another.\\
\end{enumerate}

\noindent The general limit does not exist (DNE) at $x = c$ if
\begin{enumerate}
  \item the left-hand limit does not exist at $x=c$, and/or
  \item the right-hand limit does not exist at $x=c$, and/or
  \item the left- and right-hand limits both exist, but aren't equal to one 
        another.
\end{enumerate}


% !
\section{Continuity}
\noindent \textbf{Continuity}\\
If we can draw the graph of the function without ever lifting our pencil off the
paper as we sketch it out from left to right, then the function is continuous 
everywhere. At any point where we have to lift our pencil off the paper in order
to continue sketching it, there must be a discontinuity at that point.\\\\

\noindent \textbf{Point (removable) discontinuity}\\
A point discontinuity exists wherever there's a hole in the graph at one 
specific point. These are also called “removable discontinuities” because we can
“remove” the discontinuity redefining the function at that particular point. 
Point discontinuities exist when a factor in rational functions when a factor 
that would have made the denominator 0 is cancelled from the function. The 
general limit always exists at a point discontinuity.\\\\

\noindent \textbf{Jump discontinuity}\\
A jump discontinuity exists wherever there's a big break in the graph that isn't 
caused by an asymptote. Jump discontinuities usually occur in piecewise-defined 
functions. The general limit never exists at a jump discontinuity.\\\\

\noindent \textbf{Infinite (essential) discontinuity}\\
An infinite discontinuity is the kind of discontinuity that occurs at an 
asymptote. Infinite discontinuities exist in rational functions in factors that 
make the denominator equal to 0 and can't be cancelled from the denominator.\\\\

\noindent \textbf{Endpoint discontinuity}\\
Endpoint discontinuities exist at $a$ and $b$ when a function is defined over a 
particular interval $[a, b]$. The general limit never exists at an endpoint 
discontinuity.\\\\


% !
\section{Solving Limits}

\noindent \textbf{Process for solving limits}\\
Try direct substitution first, then factoring, then conjugate method.\\\\

\noindent \textbf{Conjugate}\\
The conjugate of an expression is an expression with the same two terms, but 
with the opposite sign between the terms.\\\\

\noindent \textbf{Limits at infinity}\\
The limit at infinity is the limit of the function as we approach $\infty$ or 
$-\infty$.\\\\

\noindent \textbf{Infinite Limits}\\
A limit is infinite when the value of the limit is $\infty$ or $-\infty$ as we 
approach a particular point.\\\\

\noindent \textbf{Degree in a Rational Function}\\
The degree of the numerator or denominator is the exponent on the term with the 
largest exponent.
\begin{itemize}
  \item $N<D$: If the degree of the numerator is less than the degree of the 
               denominator, then the horizontal asymptote is given by $y=0$.
  \item $N>D$: If the degree of the numerator is greater than the degree of the 
               denominator, then the function doesn't have a horizontal 
               asymptote.
  \item $N=D$: If the degree of the numerator is equal to the degree of the 
               denominator, then the horizontal asymptote is given by the ratio 
               of the coefficients on the highest-degree terms.\\\\
\end{itemize}


% !
\section{Trignometric Limits}

Limit problems with trigonometric functions usually revolve around three key 
limit values.

\begin{equation*}
  \lim_{x \to 0} \frac{\sin x}{x} = 1
\end{equation*}

\begin{equation*}
  \lim_{x \to 0} \cos x = 1
\end{equation*}

\begin{equation*}
  \lim_{x \to 0} \sin x = 0
\end{equation*}


% !  
\section{Squeeze Theorem}
The Squeeze Theorem allows us to find the limit of a function at a particular 
point, even when the function is undefined at that point. The way that we do it 
is by showing that our function can be “squeezed” between two other functions at 
the given point, and proving that the limits of these other functions are equal.


% !
\section{Definition of the Derivative}

\noindent \textbf{Secant Line, Average Rate of Change}\\
A secant line is a line that runs right through the graph, crossing it at a 
point. The slope of the secant line is the average rate of change of the 
function over the points $(x, f(x))$ and $(x+h,f(x+h))$ at which the secant line
intersects the function.\\
\begin{equation*}
  m = \frac{f(x+h)-f(x)}{h}
\end{equation*}

\noindent \\\\\\\textbf{Tangent Line, Instantaneous Rate of Change, Difference 
                  Quotient}\\
A tangent line is a line that just barely touches the edge of a graph, 
intersecting it at exactly one specific point. The line doesn't cross the graph, 
it skims along the graph and stays along the same side of the graph. The slope 
of the tangent line is the instantaneous rate of change of the function at the 
point at which the tangent line intersects the function.\\
\begin{equation*}
  f'(x)=\lim_{h \to 0} \frac{f(x+h)-f(x)}{h}
\end{equation*}


% !
\section{Derivative Power Rules}

\begin{framed}
\noindent \textbf{Power Rule}\\
The power rule lets us take the derivative of power functions.
  \begin{equation*}
    (a \cdot n)x^{n - 1}
  \end{equation*}
\end{framed}

\begin{framed}
\noindent \textbf{Derivative of a Constant}\\
The derivative of a constant is 0\\
Given the a constant $y = 5$ for example:
  \begin{equation*}
    y = 5
  \end{equation*}
  \centerline{Since five is a constant it can be seen as:}
  \begin{equation*}
    y = 5x^0
  \end{equation*}
  \centerline{So when you use the power rule on it:}
  \begin{equation*}
    y' = (5 \cdot 0)x^{0-1}
  \end{equation*}
  \centerline{And of course anything times 0 is 0}
  \begin{equation*}
    y' = 0
  \end{equation*}
  \centerline{Thus, the derivative of a constant is 0}
\end{framed}

\begin{framed}
\noindent \textbf{Power Rule for Negative Powers}\\
When you need to take the derivative of a negative power, you can use a simple
rearrangement of terms, to make using the power rule easier.
  \begin{equation*}
    x^{-a}= \frac{1}{x^a}
  \end{equation*}
  \centerline{or}
  \begin{equation*}
    \frac{1}{x^{-a}} = x^a
  \end{equation*}
\end{framed}

\begin{framed}
\noindent \textbf{Power Rule for Fractional Powers}\\
If you need to take the derivative of the function involving a root, you can 
use a simple rearrangement of terms, to make using the power rule easier.
  \begin{equation*}
    \sqrt[b]{x^a} = x^{\frac{a}{b}}
  \end{equation*}
\end{framed}


% !
\section{Product Rule}
\noindent For $y = f(x)g(x)$, the derivative is
\begin{equation*}
  y' = f'(x)g(x) + g'(x)f(x)
\end{equation*}

\noindent \\\\For $y = f(x)g(x)h(x)$, the derivative is
\begin{equation*}
  y' = f'(x)g(x)h(x) + g'(x)f(x)h(x) + f(x)g(x)h'(x)
\end{equation*}


% !
\section{Quotient Rule}
\noindent For $y = \frac{f(x)}{g(x)}$, the derivative is:\\

\begin{equation*}
  y' = \frac{f'(x)g(x)-g'(x)f(x)}{[g(x)]^2}
\end{equation*}


% !
\section{Reciprocal Rule}
\noindent For $y = \frac{a}{g(x)}$, the derivative is:\\

\begin{equation*}
  y' = \frac{-ag'(x)}{[g(x)]^2}
\end{equation*}


% !
\section{Derivatives of the Six Trig Functions}

  \begin{center}
    \begin{tabular}{ |c|l| } 
     \hline
     \textbf{Trigonometric Function} & \textbf{Derivative} \\ 
     \hline
     $\sin(x)$ & $\cos(x)$ \\ 
     \hline
     $\cos(x)$ & $-\sin(x)$ \\ 
     \hline
     $\tan(x)$ & $\sec^2(x)$ \\ 
     \hline
     $\sec(x)$ & $\sec(x)\tan(x)$ \\ 
     \hline
     $\csc(x)$ & $-\csc(x)\cot(x)$ \\ 
     \hline
     $\cot(x)$ & $-\csc^2(x)$ \\ 
     \hline
    \end{tabular}
    \end{center}


% !
\section{Derivative of Exponential Functions}
\noindent First note that the constant $e \approx 2.718281828459045...$\\\\

\begin{center}
\begin{tabular}{|c|c|c|}
  \hline
  \textbf{Functions} & \textbf{Derivative} & \textbf{Derivative with g(x) 
  argument} \\
  \hline
  $y = e^x$ & $y'=e^x$ & $y' = e^{g(x)}g'(x)$ \\
  \hline
  $y = a^x$ & $y'=a^x(\ln a)$ & $y'=a^{g(x)}(\ln a)g'(x)$\\
  \hline
\end{tabular}
\end{center}


% !
\section{Derivative of Logarithmic Functions}
\noindent First note that the $\ln a$ is the natural logarithm, evaluated at $a$

\begin{center}
\begin{tabular}{|c|c|c|}
  \hline
  \textbf{Functions} & \textbf{Derivative} & \textbf{Derivative with g(x) 
  argument} \\
  \hline
  $y = \log_ax$ & $y'= \frac{1}{x \ln a}$ & $y' = \frac{1}{g(x)\ln a} g'(x)$ \\
  \hline
  $y = \ln x$ & $y'= \frac{1}{x}$ & $y'= \frac{1}{g(x)} g'(x)$\\
  \hline
\end{tabular}
\end{center}


% !
\section{The Chain Rule}

The chain rule is a fundamental concept in calculus that allows us to find the 
derivative of a composite function. It states that if $y = f(u)$ and $u = g(x)$, 
then the derivative of $y$ with respect to $x$ is given by:

\begin{equation}
\frac{dy}{dx} = \frac{dy}{du} \cdot \frac{du}{dx}
\end{equation}

This can be written more compactly as $dy/dx = dy/du \cdot du/dx$.

The chain rule is extremely useful in solving problems involving composite 
functions. For example, consider the function $y = (3x^2 + 4x)^5$. To find the 
derivative of this function, we can first find the derivative of the inner 
function $u = 3x^2 + 4x$:

\begin{align*}
\frac{du}{dx} &= 6x + 4 \\
\Rightarrow dy/du &= (3x^2 + 4x)^4 \\
\end{align*}

Next, we can use the chain rule to find the derivative of the outer function 
$y$ with respect to $x$:

\begin{align*}
  \frac{dy}{dx} &= \frac{dy}{du} \cdot \frac{du}{dx} \\
  &= 5(3x^2 + 4x)^4 \cdot (6x + 4) \\
  &= 5(3x^2 + 4x)^4(6x + 4)
  \end{align*}
  
  We can also use chain rule to find the derivative of multi variable function, 
  for example 
  
  $F(x,y) = x^2*sin(y)$, if we want to find $\frac{\partial F}{\partial x}$
  
  \begin{align*}
  \frac{\partial F}{\partial x} &= \frac{\partial F}{\partial x}\frac{
    \partial x}{\partial x} + \frac{\partial F}{\partial y}\frac{\partial y}
    {\partial x}\\
  &= 2x*sin(y)
  \end{align*}
  
  In summary, the chain rule is a powerful tool that allows us to find the 
  derivative of composite functions by breaking them down into simpler parts. 
  It is essential for solving many types of problems in calculus.


% !
\section{Derivatives of Inverse Trig Functions}
\begin{center}
  \begin{tabular}{|c|c|c|}
    \hline
    \textbf{Function} & \textbf{Derivative} & \textbf{With g(x) argument} \\
    \hline
    $\arcsin{x}$ & $\frac{1}{\sqrt{1-x^2}}$ & $\frac{d}{dx}(\arcsin{g(x)}) = 
    \frac{g'(x)}{\sqrt{1-g(x)^2}}$ \\
    \hline
    $\arccos{x}$ & $-\frac{1}{\sqrt{1-x^2}}$ & $\frac{d}{dx}(\arccos{g(x)}) = 
    -\frac{g'(x)}{\sqrt{1-g(x)^2}}$ \\
    \hline
    $\arctan{x}$ & $\frac{1}{1+x^2}$ & $\frac{d}{dx}(\arctan{g(x)}) = 
    \frac{g'(x)}{1+g(x)^2}$ \\
    \hline
    $\operatorname{arcsec}{x}$ & $\frac{1}{|x|\sqrt{x^2-1}}$ & 
    $\frac{d}{dx}(\operatorname{arcsec}{g(x)}) = 
    \frac{g'(x)}{|g(x)|\sqrt{g(x)^2-1}}$ \\
    \hline
    $\operatorname{arccsc}{x}$ & $-\frac{1}{|x|\sqrt{x^2-1}}$ & 
    $\frac{d}{dx}(\operatorname{arccsc}{g(x)}) = 
    -\frac{g'(x)}{|g(x)|\sqrt{g(x)^2-1}}$ \\
    \hline
    $\operatorname{arccot}{x}$ & $-\frac{1}{1+x^2}$ & 
    $\frac{d}{dx}(\operatorname{arccot}{g(x)}) = -\frac{g'(x)}{1+g(x)^2}$ \\
    \hline
    \end{tabular}
\end{center}


% !
\section{Average Rate of Change}
\begin{equation*}
  \frac{\Delta f}{\Delta x} = \frac{f(x_2)-f(x_1)}{x_2-x_1}
\end{equation*}


% !
\section{Implicit Differentiation}

Implicit differentiation is a method of finding the derivative of an implicit 
function, which is a function that is defined implicitly rather than explicitly. 
In other words, the dependent variable is not isolated on one side of the 
equation. 

Given an implicit function $F(x,y) = 0$, the derivative of $y$ with respect to 
$x$ can be found by taking the derivative of both sides of the equation with 
respect to $x$, and then solving for $\frac{dy}{dx}$. The process is summarized 
by the following formula:

$$\frac{dy}{dx} = 
-\frac{\frac{\partial F}{\partial x}}{\frac{\partial F}{\partial y}}$$

Where $\frac{\partial F}{\partial x}$ and $\frac{\partial F}{\partial y}$ are 
the partial derivatives of $F$ with respect to $x$ and $y$, respectively.

For example, consider the implicit function $x^2 + y^2 = 16$. To find 
$\frac{dy}{dx}$, we take the derivative of both sides with respect to $x$:

$$\frac{d}{dx}(x^2 + y^2) = \frac{d}{dx}(16)$$

$$2x + \frac{dy}{dx} \cdot 2y = 0$$

Solving for $\frac{dy}{dx}$ gives us:

$$\frac{dy}{dx} = -\frac{2x}{2y} = -\frac{x}{y}$$

Another example is the implicit function $x^3 + y^3 = 3xy$,

$$\frac{dy}{dx} 
= -\frac{\frac{\partial F}{\partial x}}{\frac{\partial F}{\partial y}} 
= -\frac{3x^2 - 3y^2}{3y^2 - 3x^2}$$


% !
\section{Higher Order Derivatives}
\begin{table}[h]
  \centering
  \renewcommand{\arraystretch}{1.7}
  \begin{tabular}{p{5cm}p{3cm}p{3cm}}
  \toprule
  \textbf{Higher Order Derivative} & \textbf{Form} & \textbf{Notation} \\ \hline
  First Derivative & $y'$ & $f'(x)$ or $\frac{dy}{dx}$ \\ 
  \hline
  Second Derivative & $y''$ & $f''(x)$ or $\frac{d^2y}{dx^2}$ \\ 
  \hline
  Third Derivative & $y'''$ & $f'''(x)$ or $\frac{d^3y}{dx^3}$ \\ \bottomrule
  \end{tabular}
  \end{table}
