\chapter{Calculus II}

% !
\section{Derivatives and Integral Formulas Relevant to Calculus II}

\begin{table}[h]
  \centering
  \renewcommand{\arraystretch}{2}
  \begin{tabular}{p{5cm}p{7cm}}
  \toprule
  \textbf{Differentiation Formula} 
  & \textbf{Corresponding Integration Formula}\\ \hline
  $\frac{d}{dx}x^n = nx^{n-1}$ 
  & $\int \, x^n \, dx = \frac{x^{n+1}}{n+1} + C, n \neq -1$ \\ 
  \hline
  $\frac{d}{dx} \ln x = \frac{1}{x}$ 
  & $\int \, \frac{1}{x} \, dx = \ln |x| + C$\\
  \hline
  $\frac{d}{dx} e^x = e^x$ & $\int \, e^x \, dx = e^x + C$\\
  \hline
  $\frac{d}{dx} a^x = (\ln a)a^x$ 
  & $\int \, a^x \, dx = \left(\frac{1}{\ln a}\right)a^x + C$\\
  \hline
  $\frac{d}{dx} \sin x = \cos x$ & $\int \, \cos x \, dx = \sin x + C$\\
  \hline
  $\frac{d}{dx} \cos x = -\sin x$ & $\int \, \sin x \, dx = -\cos x + C$\\
  \hline
  $\frac{d}{dx} \tan x = \sec^2 x$ & $\int \, \sec^2x \, dx = \tan x + C$\\
  \hline
  $\frac{d}{dx} \cot x = -\csc^2 x$ & $\int \, \csc^2x \, dx = -\cot x + C$\\
  \hline
  $\frac{d}{dx} \sec x = \sec x \tan x$ 
  & $\int \, \sec x \tan x \, dx = \sec x + C$\\
  \hline
  $\frac{d}{dx} \csc x = -\csc x \cot x$ 
  & $\int \, \csc x \cot x \, dx = -\csc x + C$\\
  \hline
  $\frac{d}{dx} \arcsin x = \frac{1}{\sqrt{1-x^2}}$ 
  & $\int \, \frac{1}{\sqrt{1-x^2}} \, dx = \arcsin x + C$\\
  \hline
  $\frac{d}{dx} \arccos x = -\frac{1}{\sqrt{1-x^2}}$ 
   & \\ % $\int \, -\frac{1}{\sqrt{1-x^2}} \, dx = \arccos x + C$\\
  \hline
  $\frac{d}{dx} \arctan x = \frac{1}{1+x^2}$ 
  & $\int \, \frac{1}{1+x^2} \, dx = \arctan x + C$\\
  \hline
  $\frac{d}{dx} \operatorname{arcsec} x = \frac{1}{|x|\sqrt{x^2-1}}$ 
  & $\int \, \frac{1}{|x|\sqrt{x^2-1}} \, 
  dx = \operatorname{arcsec} |x| + C$\\ \bottomrule
  \end{tabular}
  \end{table}


% !
\section{Anti-Derivatives}

Anti-derivatives, also known as indefinite integrals, are the reverse operation 
of finding the derivative of a function. Given a function $f(x)$, an 
anti-derivative of $f(x)$ is a function $F(x)$ such that $F'(x) = f(x)$. The 
symbol for anti-derivative is $\int$. 

Anti-derivatives are used in many areas of mathematics, physics, and engineering
to find the total amount of some quantity over an interval. For example, 
in physics, anti-derivatives can be used to find the distance traveled by an 
object given its velocity function. In economics, anti-derivatives can be used 
to find the total cost of producing a certain quantity of goods given the cost 
function.

The process of finding an anti-derivative of a function is called integration. 
There are several techniques for finding anti-derivatives, including integration 
by substitution, integration by parts, and using integration tables. It's 
important to note that for most functions, there is no closed-form solution for 
the antiderivative, in those cases numerical integration is used.

Here are some examples of finding anti-derivatives:

1. The anti-derivative of $x^2$ is $\frac{x^3}{3} + C$ , where C is an arbitrary 
constant.\\

2. The anti-derivative of $2x$ is $x^2 + C$ \\

3. The anti-derivative of $e^x$ is $e^x + C$


% !
\section{Limit of Riemann Sums and Definite Integrals}

\noindent \textbf{Riemann Sum:} A Riemann sum is a sum of the form:
\begin{equation*}
    \sum_{i=1}^{n} f(x_i) \Delta x
\end{equation*}
A Riemann sum is a sum of areas of $n$ rectangles with width $\Delta x$ and a
height $f(x_i)$. A Riemann sum can be used to approximate the area under the 
graph of a function $f(x)$.\\\\

\noindent \textbf{Definite Integral:} The definite integral $\int_a^b f(x)dx$ is
the exact area under the graph of a function $f(x)$ on the interval $[a, b]$.
The definite integral can be found by taking a limit of a Riemann sum as the 
number of rectangles used approaches infinity. That is,

\begin{equation*}
  \int_a^b f(x)dx = \lim_{n \to \infty} \sum_{i=1}^{n} f(x_i) \Delta x
\end{equation*}

\noindent \textbf{Steps to Rewriting the Limit of a Riemann Sum as a Definite 
                  Integral:} 
\begin{enumerate}
  \item  Determine the value of $\Delta x$. Remember that $ \int_a^b f(x)dx = 
        \lim_{n \to \infty} \sum_{i=1}^{n} f(x_i) \Delta x.$
  \item  Choose a lower bound $a=0$ and determine the upper bound $b$ using the 
         fact that $\Delta x = \frac{b-a}{n}$
  \item Determine the function $f(x)$ by replacing the $x_i = i \Delta x$ with 
        an $x$.
  \item Use the gathered information to write the definite integral 
        $\int_a^b f(x)dx$.
\end{enumerate}

\begin{framed}
  \noindent \textbf{Example:}
  Rewrite the limit of the Riemann sum as a definite integral. 
  \begin{equation*}
    \sum_{i=1}^{n} \left(5 + \frac{3i}{n}\right)^5 \cdot \frac{3}{n}
  \end{equation*}
  1. Determine the value of $\Delta x$. Remember that $ \int_a^b f(x)dx = 
        \lim_{n \to \infty} \sum_{i=1}^{n} f(x_i) \Delta x.$\\\\
  $\Delta x$ is written outside of the paraentheses in the Riemann Sum. We have:
  \begin{equation*}
    \Delta x = \frac{3}{n}
  \end{equation*}

  \noindent \\2. Choose a lower bound $a=0$ and determine the upper bound $b$ 
  using the fact that $\Delta x = \frac{b-a}{n}$\\\\

  Using $a = 0$ and $\Delta x = \frac{b - a}{n}$, we have:
  \begin{equation*}
    \Delta x = \frac{b - a}{n}
  \end{equation*}

  \begin{equation*}
    \frac{3}{n} = \frac{b - 0}{n}
  \end{equation*}

  \begin{equation*}
    \frac{3}{n} = \frac{b}{n}
  \end{equation*}

  \begin{equation*}
    3 = b
  \end{equation*}

  \noindent So we have the bounds of integration $a = 0$ and $b = 3$.\\\\
  3. Determine the function $f(x)$ by replacing the $x_i = i \Delta x$ with 
  an $x$.\\

  \noindent In the Riemann sum, we have 
  $f(x_i)=\left(5+\frac{3i}{n}\right)^5$.\\\\
  \noindent Since $\Delta x = \frac{3}{n}$, we know that $x_i=i\Delta x 
  = \frac{3i}{n}$. Therefore:

  \begin{equation*}
    f(x_i) = \left(5+\frac{3i}{n}\right)^5
  \end{equation*}

  \begin{equation*}
    f(x_i) = (5+x_i)^5
  \end{equation*}

  \begin{equation*}
    f(x_i) = (5+x)^5
  \end{equation*}

  4. item Use the gathered information to write the definite integral 
        $\int_a^b f(x)dx$.\\\\
  Using $a = 0, b = 3$ and $f(x) = (5+x)^5$, we have:\\
  \begin{equation*}
    \lim_{n \to \infty} \sum_{i=i}^{n} \left(5+\frac{3i}{n}\right)^5\cdot 
    \frac{3}{n} = \int_0^3 (5+x)^5 dx
  \end{equation*}
\end{framed}


% !
\section{U-Substitution with Examples}
u-substitution is a method for evaluating definite and indefinite integrals. 
The basic idea is to make a substitution of the variable of integration in order 
to make the integral easier to evaluate.\\\\
\textbf{3 Clues we need for u-sub}
\begin{enumerate}
  \item Product of 2 Functions
  \item One factor is a composition
  \item The other factor is the derivative of the inside factor
\end{enumerate}

\noindent Here's an example of how to use u-substitution to evaluate an 
indefinite integral:

\begin{framed}
\begin{align*}
  \int 2x \sqrt{x^2+5} \, dx\\
  u &= x^2+5\\
  du &= 2x \, dx\\
  &= \int \sqrt{u} \, du\\
  &= \int (u)^{\frac{1}{2}} \, du\\
  &= \frac{u^{\frac{3}{2}}}{\frac{3}{2}} + C\\
  &= \frac{2}{3} u^{\frac{3}{2}} + C\\
  &= \frac{2}{3} (x^2+5)^{\frac{3}{2}} + C
\end{align*}
\end{framed}

\noindent Another Example:

\begin{framed}
\begin{align*}
  \int x^2 \, e^{x^3} \, dx\\
  u &= x^3\\
  u' &= 3x^2\\
  &= \int \textcolor{red}{3\left(\frac{1}{3}\right)} \, x^2 \, e^{x^3} \, dx\\
  &\textcolor{red}{= \frac{1}{3} \int 3x^2 \, e^{x^3} \, dx}\\
  du &= 3x^2 \, dx\\
  &= \frac{1}{3} \int e^u \, du\\
  &= \frac{1}{3} e^u + C\\
  &= \frac{1}{3} e^{x^3} + C
\end{align*}
\end{framed}

\noindent Another Example:

\begin{framed}
\begin{align*}
  \int  \, \frac{y^5}{(1-y^3)^{\frac{3}{2}}} \, dy\\\\
  &= \int \, y^5 \cdot \frac{1}{(1-y^3)^{\frac{3}{2}}} \, dy\\\\
  u &= 1-y^3 \; \; \; \text{  so, }  \; \; \; y^3 = 1-u\\
  du &= -3y^2dy\\
  -\frac{1}{3}du &= y^2dy\\\\
  &= \int \left(y^2 \cdot y^3\right) \cdot \frac{1}{(1-y^3)^{\frac{3}{2}}} \, dy\\
  &= -\frac{1}{3} \int (1-u) \cdot \frac{1}{u^{\frac{3}{2}}} \, du\\
  &= -\frac{1}{3} \int (1-u) \cdot \left(u^{-\frac{3}{2}}\right) \, du\\
  &= -\frac{1}{3} \int u^{-\frac{3}{2}} - u^{-\frac{1}{2}} \, du\\
  &= -\frac{1}{3} \left[\frac{u^{-\frac{1}{2}}}{-\frac{1}{2}} 
  - \frac{u^{\frac{1}{2}}}{\frac{1}{2}}\right] + C\\
  &= -\frac{1}{3} \left[-2u^{-\frac{1}{2}} - 2u^{\frac{1}{2}}\right] + C\\
  &= \frac{2u^{-\frac{1}{2}}}{3} + \frac{2u^{\frac{1}{2}}}{3} + C\\
  &= \frac{2}{3u^{\frac{1}{2}}} + \frac{2u^{\frac{1}{2}}}{3} + C\\
  &= \frac{2}{3\sqrt{u}} + \frac{2\sqrt{u}}{3} + C\\
  &= \frac{2}{3\sqrt{1-y^3}} + \frac{2\sqrt{1-y^3}}{3} + C
\end{align*}
\end{framed}

\newpage

% !
\section{U-Substitution with Logs}
\noindent When using logs, the substitution is typically made with the variable 
$u = \ln(x)$. This is because the derivative of $\ln(x)$ is $\frac{1}{x}$, which 
makes it easy to integrate when the function contains $x$ in the denominator. 
By making this substitution, the integral can be rewritten in terms of $u$, 
making it easier to evaluate.\\

\noindent It's important to remember that when making a substitution with logs, 
you need to change the limits of integration as well, and you need to take the 
absolute value of u when evaluating the integral.\\

\noindent For example:

\begin{framed}
\begin{align*}
  \int_{e^2}^{e^3} \frac{\ln x}{x} \, dx\\
  &= \int_{e^2}^{e^3} \ln x \cdot \frac{1}{x} \, dx\\
  u &= \ln x\\
  u' &= \frac{1}{x}\\
  du &= \frac{1}{x} \, dx\\
  &= \int_{u(e^2)}^{u(e^3)} u \, du\\\\
  u(e^3) &= ln(e^3) = 3\\
  u(e^2) &= ln(e^2) = 2\\\\
  \int_{2}^{3} u \, du &= \frac{u^2}{2} \biggr\rvert_{2}^{3}\\
  &= \frac{(3)^2}{2} - \frac{(2)^2}{2}\\
  &= \frac{9}{2} - \frac{4}{2}\\
  &= \frac{5}{2}
\end{align*}
\end{framed}

\newpage

\noindent Another example:

\begin{framed}
  Here $\frac{1}{x \cos x}$ is a function composition because of
  the reciprocal function. Meaning we can try this for $u$. That then means
  we need to use the product rule to find $du$.
\begin{align*}
  \int \frac{\cos x - x \sin x}{x\cos x} \, dx\\
  &= \int \frac{1}{x\cos x} \cdot \left(\cos x - x\sin x\right) \, dx\\\\
  u &= x\cos x\\
  u' &= 1 \cdot \cos x + (- \sin x) \cdot x\\
  du &= \left(\cos x - x \sin x\right) \, dx\\\\
  &= \int \frac{1}{u} \, du\\
  &= \ln |u| + C\\
  &= \ln |x\cos x| + C
\end{align*}
\end{framed}


\newpage

% !
\section{U-Substitution with Inverse Trig}
\noindent u-substitution can also be used to evaluate integrals involving 
inverse trigonometric functions. Here's an example:

\begin{framed}
  \begin{align*}
    \int \frac{1}{4+x^2} \, dx\\
    &= \int \frac{1}{4\left(1+\frac{x^2}{4}\right)} \, dx\\
    &= \frac{1}{4} \int \frac{1}{1 + \left(\frac{x}{2}\right)^2} \, dx\\\\
    u&=\frac{x}{2}\\
    u'&=\frac{1}{2}\\\\
    &= \frac{1}{4} \textcolor{red}{(\cdot 2)} \int 
    \textcolor{red}{\left(\frac{1}{2}\right) \cdot }\frac{1}{1 
    + \left(\frac{x}{2}\right)^2} \, dx\\\\
    du &= \frac{1}{2} \, dx\\
    &= \frac{1}{2} \int \frac{1}{1+u^2} \, du\\
    &= \frac{1}{2} \operatorname{arctan} (u) + C\\
    &= \frac{1}{2} \operatorname{arctan} \left(\frac{x}{2}\right) + C
    \end{align*}
\end{framed}

!
\section{Area Between Curves}
In Calculus 2, the area between two curves refers to the region enclosed by two 
functions and the x-axis on a coordinate plane. The area between the curves can 
be found by subtracting the area under one function (the "bottom" function) from 
the area under the other function (the "top" function).\\

One way to find the area between the curves is by using definite integrals. The 
definite integral of a function gives the signed area under the curve of that 
function. The definite integral of a function over an interval [a, b] gives the 
signed area between the curve of that function and the x-axis over that 
interval. To find the area between two curves, we can find the definite integral 
of each function over the same interval, and then subtract the definite integral 
of the "bottom" function from the definite integral of the "top" function.\\

For example, if we have two functions f(x) and g(x), and we want to find the 
area between the curves for the interval [a, b], we can use the following 
formula: Area = $\int_{a}^{b}$ [f(x) - g(x)] dx\\

It is important to keep in mind that the definite integral gives the signed 
area, so if the "top" function is below the "bottom" function over the interval, 
the area between the curves will be negative.\\

Another way to find the area between the curves is by using Riemann Sums, which 
is a method that approximates the area between the curves by dividing the 
interval into smaller subintervals, and computing the area of rectangles with 
heights at the points of the function at the right endpoint of each 
subinterval.\\

It is important to note that when finding the area between curves, it is 
important to check the function and make sure that the area we are trying to 
find is the area between the curves and not the area of one of the functions.\\

In some cases, the area between the curves might not be continuous, in those 
cases we will have to split the region into smaller regions and find the area 
for each of them separately.\\

\begin{framed}
  \begin{align*}
    \text{Area} &= \int_{left}^{right} (\text{top} - \text{bottom}) \, dx\\\\
    \text{Area} &= \int_{bottom}^{top} (\text{right} - \text{left}) \, dy
  \end{align*}
\end{framed}


% !
\section{Volume by Slicing}

\begin{framed}
  \noindent \textbf{Disks}
  \begin{align*}
    \text{x-axis} &= \int_a^b \pi [f(x)]^2 \, dx\\\\
    \text{y-axis} &= \int_c^d \pi [f(y)]^2 \, dy
  \end{align*}
\end{framed}

\begin{framed}
  \noindent \textbf{Washers/Donuts}
  \begin{align*}
    \text{x-axis} &= \int_a^b \pi [f(x)]^2 - \pi [g(x)]^2 \, dx\\
    \text{When } y=-k, \; &= \int_a^b \pi [k+f(x)]^2 - \pi [k+g(x)]^2 \, dx\\
    \text{When } y=k, \; &= \int_a^b \pi [k-f(x)]^2 - \pi [k-g(x)]^2 \, dx\\\\\\
    \text{y-axis} &= \int_c^d \pi [f(y)]^2 - \pi [g(y)]^2 \, dy\\
    \text{When } x=-k, \; &= \int_c^d \pi [k+f(y)]^2 - \pi [k+g(y)]^2 \, dy\\
    \text{When } x=k, \; &= \int_c^d \pi [k-f(y)]^2 - \pi [k-g(y)]^2 \, dy
  \end{align*}
\end{framed}


% !
\section{Volume of Revolution By Cylindrical Shells}

\begin{equation*}
  \int \text{\textcolor{cyan}{circumference}} \;
       \text{\textcolor{teal}{height/width}} \;
       \text{\textcolor{orange}{thickness}}
\end{equation*}

\begin{framed}
  \noindent \textbf{Shells}
  \begin{align*}
    \text{x-axis} &= \int_c^d \textcolor{cyan}{2\pi y} \,
    \textcolor{teal}{[f(y)-g(y)]} \, \textcolor{orange}{dy}\\
    \text{When } y=-k, \; &= \int_c^d \textcolor{cyan}{2\pi (y+k)} \,
    \textcolor{teal}{[f(y)-g(y)]} \, \textcolor{orange}{dy}\\
    \text{When } y=k, \; &= \int_c^d \textcolor{cyan}{2\pi (k-y)} \,
    \textcolor{teal}{[f(y)-g(y)]} \, \textcolor{orange}{dy}\\\\\\
    \text{y-axis} &= \int_a^b \textcolor{cyan}{2\pi x} \,
    \textcolor{teal}{[f(x)-g(x)]} \, \textcolor{orange}{dx}\\
    \text{When } x=-k, \; &= \int_a^b \textcolor{cyan}{2\pi (x+k)} \,
    \textcolor{teal}{[f(x)-g(x)]} \, \textcolor{orange}{dx}\\
    \text{When } x=k, \; &= \int_a^b \textcolor{cyan}{2\pi (k-x)} \,
    \textcolor{teal}{[f(x)-g(x)]} \, \textcolor{orange}{dx}
  \end{align*}
\end{framed}



% !
\section{Arc Length of a Curve}

\begin{framed}
  \begin{align*}
    \text{Arc Length } &= \int_a^b \, \sqrt{1 + \left[f'(x)\right]^2} \; dx\\\\
    \text{Arc Length } &= \int_c^d \, \sqrt{1 + \left[f'(y)\right]^2} \; dy\\\\\\\\
    \text{Riemann sum version} &= \lim_{n \to \infty} \sum_{i=1}^{n} 
    \sqrt{1 + \left[f'(x_{i}^{*})\right]^2} \; \Delta x 
  \end{align*}
  Where, $x \in [a, b].$ and $y \in [c, d].$
\end{framed}