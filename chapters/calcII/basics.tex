\chapter{Calculus II}

% !
\section{Derivatives and Integral Formulas Relevant to Calculus II}

\begin{table}[h]
  \centering
  \renewcommand{\arraystretch}{2}
  \begin{tabular}{p{5cm}p{7cm}}
  \toprule
  \textbf{Differentiation Formula} 
  & \textbf{Corresponding Integration Formula}\\ \hline
  $\frac{d}{dx}x^n = nx^{n-1}$ 
  & $\int \, x^n \, dx = \frac{x^{n+1}}{n+1} + C, n \neq -1$ \\ 
  \hline
  $\frac{d}{dx} \ln x = \frac{1}{x}$ 
  & $\int \, \frac{1}{x} \, dx = \ln |x| + C$\\
  \hline
  $\frac{d}{dx} e^x = e^x$ & $\int \, e^x \, dx = e^x + C$\\
  \hline
  $\frac{d}{dx} a^x = (\ln a)a^x$ 
  & $\int \, a^x \, dx = \left(\frac{1}{\ln a}\right)a^x + C$\\
  \hline
  $\frac{d}{dx} \sin x = \cos x$ & $\int \, \cos x \, dx = \sin x + C$\\
  \hline
  $\frac{d}{dx} \cos x = -\sin x$ & $\int \, \sin x \, dx = -\cos x + C$\\
  \hline
  $\frac{d}{dx} \tan x = \sec^2 x$ & $\int \, \sec^2x \, dx = \tan x + C$\\
  \hline
  $\frac{d}{dx} \cot x = -\csc^2 x$ & $\int \, \csc^2x \, dx = -\cot x + C$\\
  \hline
  $\frac{d}{dx} \sec x = \sec x \tan x$ 
  & $\int \, \sec x \tan x \, dx = \sec x + C$\\
  \hline
  $\frac{d}{dx} \csc x = -\csc x \cot x$ 
  & $\int \, \csc x \cot x \, dx = -\csc x + C$\\
  \hline
  $\frac{d}{dx} \arcsin x = \frac{1}{\sqrt{1-x^2}}$ 
  & $\int \, \frac{1}{\sqrt{1-x^2}} \, dx = \arcsin x + C$\\
  \hline
  $\frac{d}{dx} \arccos x = -\frac{1}{\sqrt{1-x^2}}$ 
   & \\ % $\int \, -\frac{1}{\sqrt{1-x^2}} \, dx = \arccos x + C$\\
  \hline
  $\frac{d}{dx} \arctan x = \frac{1}{1+x^2}$ 
  & $\int \, \frac{1}{1+x^2} \, dx = \arctan x + C$\\
  \hline
  $\frac{d}{dx} \operatorname{arcsec} x = \frac{1}{|x|\sqrt{x^2-1}}$ 
  & $\int \, \frac{1}{|x|\sqrt{x^2-1}} \, 
  dx = \operatorname{arcsec} |x| + C$\\ \bottomrule
  \end{tabular}
  \end{table}


% !
\section{Anti-Derivatives}

Anti-derivatives, also known as indefinite integrals, are the reverse operation 
of finding the derivative of a function. Given a function $f(x)$, an 
anti-derivative of $f(x)$ is a function $F(x)$ such that $F'(x) = f(x)$. The 
symbol for anti-derivative is $\int$. 

Anti-derivatives are used in many areas of mathematics, physics, and engineering
to find the total amount of some quantity over an interval. For example, 
in physics, anti-derivatives can be used to find the distance traveled by an 
object given its velocity function. In economics, anti-derivatives can be used 
to find the total cost of producing a certain quantity of goods given the cost 
function.

The process of finding an anti-derivative of a function is called integration. 
There are several techniques for finding anti-derivatives, including integration 
by substitution, integration by parts, and using integration tables. It's 
important to note that for most functions, there is no closed-form solution for 
the antiderivative, in those cases numerical integration is used.

Here are some examples of finding anti-derivatives:

1. The anti-derivative of $x^2$ is $\frac{x^3}{3} + C$ , where C is an arbitrary 
constant.\\

2. The anti-derivative of $2x$ is $x^2 + C$ \\

3. The anti-derivative of $e^x$ is $e^x + C$


% !
\section{Limit of Riemann Sums and Definite Integrals}

\noindent \textbf{Riemann Sum:} A Riemann sum is a sum of the form:
\begin{equation*}
    \sum_{i=1}^{n} f(x_i) \Delta x
\end{equation*}
A Riemann sum is a sum of areas of $n$ rectangles with width $\Delta x$ and a
height $f(x_i)$. A Riemann sum can be used to approximate the area under the 
graph of a function $f(x)$.\\\\

\noindent \textbf{Definite Integral:} The definite integral $\int_a^b f(x)dx$ is
the exact area under the graph of a function $f(x)$ on the interval $[a, b]$.
The definite integral can be found by taking a limit of a Riemann sum as the 
number of rectangles used approaches infinity. That is,

\begin{equation*}
  \int_a^b f(x)dx = \lim_{n \to \infty} \sum_{i=1}^{n} f(x_i) \Delta x
\end{equation*}

\noindent \textbf{Steps to Rewriting the Limit of a Riemann Sum as a Definite 
                  Integral:} 
\begin{enumerate}
  \item  Determine the value of $\Delta x$. Remember that $ \int_a^b f(x)dx = 
        \lim_{n \to \infty} \sum_{i=1}^{n} f(x_i) \Delta x.$
  \item  Choose a lower bound $a=0$ and determine the upper bound $b$ using the 
         fact that $\Delta x = \frac{b-a}{n}$
  \item Determine the function $f(x)$ by replacing the $x_i = i \Delta x$ with 
        an $x$.
  \item Use the gathered information to write the definite integral 
        $\int_a^b f(x)dx$.
\end{enumerate}

\begin{framed}
  \noindent \textbf{Example:}
  Rewrite the limit of the Riemann sum as a definite integral. 
  \begin{equation*}
    \sum_{i=1}^{n} \left(5 + \frac{3i}{n}\right)^5 \cdot \frac{3}{n}
  \end{equation*}
  1. Determine the value of $\Delta x$. Remember that $ \int_a^b f(x)dx = 
        \lim_{n \to \infty} \sum_{i=1}^{n} f(x_i) \Delta x.$\\\\
  $\Delta x$ is written outside of the paraentheses in the Riemann Sum. We have:
  \begin{equation*}
    \Delta x = \frac{3}{n}
  \end{equation*}

  \noindent \\2. Choose a lower bound $a=0$ and determine the upper bound $b$ 
  using the fact that $\Delta x = \frac{b-a}{n}$\\\\

  Using $a = 0$ and $\Delta x = \frac{b - a}{n}$, we have:
  \begin{equation*}
    \Delta x = \frac{b - a}{n}
  \end{equation*}

  \begin{equation*}
    \frac{3}{n} = \frac{b - 0}{n}
  \end{equation*}

  \begin{equation*}
    \frac{3}{n} = \frac{b}{n}
  \end{equation*}

  \begin{equation*}
    3 = b
  \end{equation*}

  \noindent So we have the bounds of integration $a = 0$ and $b = 3$.\\\\
  3. Determine the function $f(x)$ by replacing the $x_i = i \Delta x$ with 
  an $x$.\\

  \noindent In the Riemann sum, we have 
  $f(x_i)=\left(5+\frac{3i}{n}\right)^5$.\\\\
  \noindent Since $\Delta x = \frac{3}{n}$, we know that $x_i=i\Delta x 
  = \frac{3i}{n}$. Therefore:

  \begin{equation*}
    f(x_i) = \left(5+\frac{3i}{n}\right)^5
  \end{equation*}

  \begin{equation*}
    f(x_i) = (5+x_i)^5
  \end{equation*}

  \begin{equation*}
    f(x_i) = (5+x)^5
  \end{equation*}

  4. item Use the gathered information to write the definite integral 
        $\int_a^b f(x)dx$.\\\\
  Using $a = 0, b = 3$ and $f(x) = (5+x)^5$, we have:\\
  \begin{equation*}
    \lim_{n \to \infty} \sum_{i=i}^{n} \left(5+\frac{3i}{n}\right)^5\cdot 
    \frac{3}{n} = \int_0^3 (5+x)^5 dx
  \end{equation*}
\end{framed}