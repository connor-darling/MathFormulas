\section{Work}

\noindent The most basic form of work that can be described by math is:
``A constant force applied in the direction in which an object moves.''

\begin{equation*}
  \text{Work} = (\text{Force})(\text{Distance})
\end{equation*}

\vspace{0.3in}

For example, let's say I apply a force of 10 lbs to move an object
3 ft. In this case:

\begin{equation*}
  \text{Work} = (10 \; lbs)(3 \; ft) = 30 \; ft-lbs
\end{equation*}


Here we see that $\text{Units}=(\text{Units of Force})(\text{Units of Distance})$
which is why we end up with $ft-lbs$. Other common units are $N-m$ which come 
from Newton-meters

\noindent The general rule of thumb when dealing with work problems involving pumps is:

\begin{framed}
  \begin{itemize} 
    \item We will call where the work begins: $\phi$
    \item We will call where the work ends: $\Theta$
    \item And in most cases weight denisty is a constant, but we'll call it $\rho$
    \item And for the volume, in most cases it involve a varying function;
              for the sake of example we will call the cross sections circular
  \end{itemize} 
\begin{align*}
  \text{Work} = \int_{\phi}^{\Theta}  \, (\text{force})\cdot(\text{distance})\\
  &= \int_{\phi}^{\Theta} \, (\text{weight})\cdot(\text{x})\\
  &= \int_{\phi}^{\Theta} \, (\rho)\cdot(\text{volume})\cdot(\text{x})\\
  &= \int_{\phi}^{\Theta} \, (\rho)\cdot[\pi(r(x))^2 \,dx]\cdot(\text{x})\\
\end{align*}
\end{framed}

Along with pump problems, are spring problems. This is typically where we 
use \textbf{Hooke's Law} which states: 
\noindent \underline{The force required to hold a 
spring $x$ units from its rest position is \textbf{proportional} to the distance
$x$.}\\

\noindent In this context, proportional means ``related by a constant.''

\begin{framed}
  \begin{itemize} 
    \item $k$ is the spring constant
  \end{itemize} 
\begin{align*}
  F(x) = kx
\end{align*}
\end{framed}

\noindent Example of hooke's law:

\begin{framed}
    \noindent Problem: It takes $5N$ of force to hold a spring $3m$ from rest.\\
    a.) Find the spring constant $k$.\\
    b.) Find the work required to move the spring and additional $2m$ away from
        its rest position.
\begin{align*}
  F(x) = kx\\\\
  5 &= k \cdot 3\\
  k &= \frac{5}{3}\\
  F(x) &= \frac{5}{3} \cdot x\\\\
  \text{Work} &= \int_3^5 \, \frac{5}{3} \cdot x \, dx\\
  &= \frac{5}{3} \int_3^5 \, x dx\\
  &= . . .\\
  &= \frac{40}{3} N \cdot m
\end{align*}
\end{framed}