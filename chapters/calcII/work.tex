\section{Work}

\noindent The most basic form of work that can be described by math is:
``A constant force applied in the direction in which an object moves.''

\begin{equation*}
  \text{Work} = (\text{Force})(\text{Distance})
\end{equation*}

\vspace{0.3in}

For example, let's say I apply a force of 10 lbs to move an object
3 ft. In this case:

\begin{equation*}
  \text{Work} = (10 \; lbs)(3 \; ft) = 30 \; ft-lbs
\end{equation*}


Here we see that $\text{Units}=(\text{Units of Force})(\text{Units of Distance})$
which is why we end up with $ft-lbs$. Other common units are $N-m$ which come 
from Newton-meters