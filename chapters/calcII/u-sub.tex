% !
\section{U-Substitution with Examples}
u-substitution is a method for evaluating definite and indefinite integrals. 
The basic idea is to make a substitution of the variable of integration in order 
to make the integral easier to evaluate.\\\\
\textbf{3 Clues we need for u-sub}
\begin{enumerate}
  \item Product of 2 Functions
  \item One factor is a composition
  \item The other factor is the derivative of the inside factor
\end{enumerate}

\noindent Here's an example of how to use u-substitution to evaluate an 
indefinite integral:

\begin{framed}
\begin{align*}
  \int 2x \sqrt{x^2+5} \, dx\\
  u &= x^2+5\\
  du &= 2x \, dx\\
  &= \int \sqrt{u} \, du\\
  &= \int (u)^{\frac{1}{2}} \, du\\
  &= \frac{u^{\frac{3}{2}}}{\frac{3}{2}} + C\\
  &= \frac{2}{3} u^{\frac{3}{2}} + C\\
  &= \frac{2}{3} (x^2+5)^{\frac{3}{2}} + C
\end{align*}
\end{framed}

\noindent Another Example:

\begin{framed}
\begin{align*}
  \int x^2 \, e^{x^3} \, dx\\
  u &= x^3\\
  u' &= 3x^2\\
  &= \int \textcolor{red}{3\left(\frac{1}{3}\right)} \, x^2 \, e^{x^3} \, dx\\
  &\textcolor{red}{= \frac{1}{3} \int 3x^2 \, e^{x^3} \, dx}\\
  du &= 3x^2 \, dx\\
  &= \frac{1}{3} \int e^u \, du\\
  &= \frac{1}{3} e^u + C\\
  &= \frac{1}{3} e^{x^3} + C
\end{align*}
\end{framed}

\noindent Another Example:

\begin{framed}
\begin{align*}
  \int  \, \frac{y^5}{(1-y^3)^{\frac{3}{2}}} \, dy\\\\
  &= \int \, y^5 \cdot \frac{1}{(1-y^3)^{\frac{3}{2}}} \, dy\\\\
  u &= 1-y^3 \; \; \; \text{  so, }  \; \; \; y^3 = 1-u\\
  du &= -3y^2dy\\
  -\frac{1}{3}du &= y^2dy\\\\
  &= \int \left(y^2 \cdot y^3\right) \cdot \frac{1}{(1-y^3)^{\frac{3}{2}}} \, dy\\
  &= -\frac{1}{3} \int (1-u) \cdot \frac{1}{u^{\frac{3}{2}}} \, du\\
  &= -\frac{1}{3} \int (1-u) \cdot \left(u^{-\frac{3}{2}}\right) \, du\\
  &= -\frac{1}{3} \int u^{-\frac{3}{2}} - u^{-\frac{1}{2}} \, du\\
  &= -\frac{1}{3} \left[\frac{u^{-\frac{1}{2}}}{-\frac{1}{2}} 
  - \frac{u^{\frac{1}{2}}}{\frac{1}{2}}\right] + C\\
  &= -\frac{1}{3} \left[-2u^{-\frac{1}{2}} - 2u^{\frac{1}{2}}\right] + C\\
  &= \frac{2u^{-\frac{1}{2}}}{3} + \frac{2u^{\frac{1}{2}}}{3} + C\\
  &= \frac{2}{3u^{\frac{1}{2}}} + \frac{2u^{\frac{1}{2}}}{3} + C\\
  &= \frac{2}{3\sqrt{u}} + \frac{2\sqrt{u}}{3} + C\\
  &= \frac{2}{3\sqrt{1-y^3}} + \frac{2\sqrt{1-y^3}}{3} + C
\end{align*}
\end{framed}

\newpage

% !
\section{U-Substitution with Logs}
\noindent When using logs, the substitution is typically made with the variable 
$u = \ln(x)$. This is because the derivative of $\ln(x)$ is $\frac{1}{x}$, which 
makes it easy to integrate when the function contains $x$ in the denominator. 
By making this substitution, the integral can be rewritten in terms of $u$, 
making it easier to evaluate.\\

\noindent It's important to remember that when making a substitution with logs, 
you need to change the limits of integration as well, and you need to take the 
absolute value of u when evaluating the integral.\\

\noindent For example:

\begin{framed}
\begin{align*}
  \int_{e^2}^{e^3} \frac{\ln x}{x} \, dx\\
  &= \int_{e^2}^{e^3} \ln x \cdot \frac{1}{x} \, dx\\
  u &= \ln x\\
  u' &= \frac{1}{x}\\
  du &= \frac{1}{x} \, dx\\
  &= \int_{u(e^2)}^{u(e^3)} u \, du\\\\
  u(e^3) &= ln(e^3) = 3\\
  u(e^2) &= ln(e^2) = 2\\\\
  \int_{2}^{3} u \, du &= \frac{u^2}{2} \biggr\rvert_{2}^{3}\\
  &= \frac{(3)^2}{2} - \frac{(2)^2}{2}\\
  &= \frac{9}{2} - \frac{4}{2}\\
  &= \frac{5}{2}
\end{align*}
\end{framed}

\newpage

\noindent Another example:

\begin{framed}
  Here $\frac{1}{x \cos x}$ is a function composition because of
  the reciprocal function. Meaning we can try this for $u$. That then means
  we need to use the product rule to find $du$.
\begin{align*}
  \int \frac{\cos x - x \sin x}{x\cos x} \, dx\\
  &= \int \frac{1}{x\cos x} \cdot \left(\cos x - x\sin x\right) \, dx\\\\
  u &= x\cos x\\
  u' &= 1 \cdot \cos x + (- \sin x) \cdot x\\
  du &= \left(\cos x - x \sin x\right) \, dx\\\\
  &= \int \frac{1}{u} \, du\\
  &= \ln |u| + C\\
  &= \ln |x\cos x| + C
\end{align*}
\end{framed}


\newpage

% !
\section{U-Substitution with Inverse Trig}
\noindent u-substitution can also be used to evaluate integrals involving 
inverse trigonometric functions. Here's an example:

\begin{framed}
  \begin{align*}
    \int \frac{1}{4+x^2} \, dx\\
    &= \int \frac{1}{4\left(1+\frac{x^2}{4}\right)} \, dx\\
    &= \frac{1}{4} \int \frac{1}{1 + \left(\frac{x}{2}\right)^2} \, dx\\\\
    u&=\frac{x}{2}\\
    u'&=\frac{1}{2}\\\\
    &= \frac{1}{4} \textcolor{red}{(\cdot 2)} \int 
    \textcolor{red}{\left(\frac{1}{2}\right) \cdot }\frac{1}{1 
    + \left(\frac{x}{2}\right)^2} \, dx\\\\
    du &= \frac{1}{2} \, dx\\
    &= \frac{1}{2} \int \frac{1}{1+u^2} \, du\\
    &= \frac{1}{2} \operatorname{arctan} (u) + C\\
    &= \frac{1}{2} \operatorname{arctan} \left(\frac{x}{2}\right) + C
    \end{align*}
\end{framed}