\chapter{Pre-Calculus}

% !
\section{Composition of Functions}
\begin{minipage}{0.4\textwidth}
  \begin{equation*}
    (f \circ g)(x) = f(g(x))
  \end{equation*}

  \begin{equation*}
    (g \circ f)(x) = g(f(x))
  \end{equation*}
\vspace{0.5cm}
\end{minipage}


% !
\section{Difference Quotient}

\noindent This formula computes the slope of the secant line through two points 
on the graph of f. These are the points with x-coordinates x and x + h. The 
difference quotient is used in the definition the derivative. The formula is:

\begin{equation*}
  \frac{f(x+h)-f(x)}{h}
\end{equation*}


% !
\section{Simple Interest Formula}

\noindent $I$ = Interest\\
$P$ = Principal\\
$t$ = Time in years\\
$r$ = Annual interest rate

\begin{equation*}
  I=Prt
\end{equation*}


% !
\section{Compounding Interest Formula}

\noindent $A$ = final amount\\
$P$ = initial principal balance\\
$r$ = interest rate\\
$n$ = number of times interest applied per time period\\
$t$ = number of time periods elapsed

\begin{equation*}
  A = P(1 + \frac{r}{n})^{nt}
\end{equation*}

\newpage

% !
\section{Continuous Compounding Interest}

\noindent $P(t)$ = value at time t\\
$P_0$ = original principal sum\\
$r$ = nominal annual interest rate\\
$t$ = length of time the interest is applied

\begin{equation*}
  P(t) = P_0 e^{rt}
\end{equation*}